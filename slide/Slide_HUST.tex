\documentclass[t, aspectratio=169]{beamer} % Khai báo tỉ lệ cho Beamer biết
\usepackage[utf8]{vietnam}
\usepackage{amsthm,amsmath,amssymb}
\usepackage{lmodern}
%\usepackage{times} % Đổi font chữ sang Times New Roman (không khuyến khích vì font xấu quá)

% --- CẤU HÌNH LỀ (Phải đặt ở đây) ---
\def\slideContentLeftMargin{0.5cm}  % Lề trái
\def\slideContentRightMargin{1cm}   % Lề phải
\def\hustHeaderHeight{1cm}        % Lề trên (Né thanh tiêu đề)
\def\slideContentBotMargin{3.5cm}   % Lề dưới (Né logo/footer)


% Thay đổi [red] thành [blue] để đổi màu
% Thay đổi [169] thành [43] (nhớ đổi cả trong documentclass) để đổi tỉ lệ
\usepackage[red, 169]{beamerthemeHUST} 

% 1. Điều chỉnh vị trí cả khối title ở trang title
\renewcommand{\titlePageTitleX}{-0.5cm}
\renewcommand{\titlePageTitleY}{-0.5cm}

% 2. Căn lề TỪNG THÀNH PHẦN của trang title
% - Tiêu đề chính: Căn TRÁI
\renewcommand{\hustTitleAlign}{\alignLeft} 

% - Phụ đề: Căn TRÁI
\renewcommand{\hustSubtitleAlign}{\alignLeft}

% - Tác giả & Trường: Căn TRÁI
\renewcommand{\hustAuthorAlign}{\alignLeft}
\renewcommand{\hustInstAlign}{\alignLeft}


% Bạn có thể dùng \\ để xuống dòng thủ công trong phần title
\title{NGHIÊN CỨU VÀ ỨNG DỤNG MẠNG CNN TRONG NHẬN DIỆN ẢNH Y TẾ}
%\subtitle{Báo cáo Đồ án tốt nghiệp - Nhóm FRC}
% Bạn có thể dùng \\ để xuống dòng thủ công trong phần author
\author{\texorpdfstring{Nguyễn Văn A - 2023xxxx \\ Trịnh Thị B - 2024xxxxx}{Nguyễn Văn A - 2023xxxx, Trịnh Thị B - 2024xxxxx}} 
\institute{FRC - FaMI Rebel Club}
%\date{\today}

\begin{document}
	
	\hustintropage{} % Có thể comment trang này để tắt slide đi, nếu không muốn sử dụng.
	\hustsectionpage{} % Có thể comment trang này để tắt slide đi, nếu không muốn sử dụng.
	
	
	
	
	% Trang bìa
	\husttitlepage
	
	\begin{frame}[allowframebreaks]{Nội dung trình bày}
		\tableofcontents %Mục lục slide, sử dụng \tableofcontents[hideallsubsections] nếu mục lục quá dài, chỉ cần hiện các mục lớn.
	\end{frame}
	
	\resetPageCount  % Đặt lại bộ đếm để trang Title là trang 1
	\showPageNumber  % Bật hiển thị số trang (Sẽ hiện mãi mãi về sau)
	
	% =================================================================
	% Có 6 lệnh tùy chỉnh slide, gồm \useThinBar và \useThickBar cho thanh kẻ slide trên cùng dày/mỏng và \useStyleFull, \useStyleLogoOnly, \useStyleHustFull, \useStyleHustOnly để đổi logo góc phải bên dưới. 
	
	%Mặc định khai báo của slide là \useThinBar và \useStyleFull.
	% =================================================================
	
	
	% =================================================================
	% PHẦN NỘI DUNG 1: SỬ DỤNG STYLE MẶC ĐỊNH 
	% =================================================================
	\useStyleHustFull
	\useThinBar 
	
	\section{Giới thiệu bài toán}
	
	\begin{frame}{1. Đặt vấn đề}
		\textbf{Bối cảnh thực tế:}
		\begin{itemize}
			\item Nhu cầu chẩn đoán hình ảnh tự động ngày càng cao trong y tế thông minh.
			\item Các phương pháp truyền thống tốn nhiều thời gian và phụ thuộc vào kinh nghiệm bác sĩ.
			\item \textbf{Mục tiêu:} Xây dựng mô hình AI hỗ trợ chẩn đoán với độ chính xác $>95\%$.
		\end{itemize}
		
		\vspace{1em}
		
		\textbf{Thách thức kỹ thuật:}
		\begin{itemize}
			\item Dữ liệu ảnh y tế thường bị nhiễu và mất cân bằng.
			\item Yêu cầu khả năng giải thích (Explainable AI) cao.
			\item \alert{Ràng buộc:} Phải chạy được trên các thiết bị Edge Device tại bệnh viện.
		\end{itemize}
	\end{frame}
	
	% =================================================================
	% PHẦN NỘI DUNG 2: TOÁN HỌC (DEMO PAUSE & CÔNG THỨC)
	% =================================================================
	\section{Cơ sở lý thuyết}
	
	\begin{frame}{2. Phép tích chập (Convolution)}
		Mạng CNN hoạt động dựa trên phép toán tích chập giữa ma trận ảnh đầu vào $I$ và bộ lọc (kernel) $K$.
		
		\pause % Dừng lại để thuyết trình, số trang sẽ tăng lên.
		
		\textbf{Công thức toán học:}
		Cho ảnh đầu vào $I$ kích thước $H \times W$ và bộ lọc $K$ kích thước $k \times k$. Giá trị tại vị trí $(i, j)$ của bản đồ đặc trưng (feature map) $S$ được tính như sau:
		
		\begin{equation}
			S(i, j) = (I * K)(i, j) = \sum_{m=0}^{k-1} \sum_{n=0}^{k-1} I(i+m, j+n) \cdot K(m, n)
		\end{equation}
		
		\pause % Dừng lần nữa
		
		\textbf{Ý nghĩa:}
		\begin{itemize}
			\item Giúp trích xuất các đặc trưng cục bộ (cạnh, góc, vân bề mặt).
			\item Giảm số lượng tham số cần huấn luyện so với mạng nơ-ron truyền thống (Fully Connected).
		\end{itemize}
 	\end{frame}
	
	% =================================================================
	% PHẦN NỘI DUNG 3: CHIA CỘT & ĐỔI GIAO DIỆN 
	% =================================================================
	\useStyleLogoOnly 
	\useThickBar      
	
	\section{Kiến trúc đề xuất}
	
	\begin{frame}{3. Mô hình ResNet-50 cải tiến}
		Chúng tôi đề xuất sử dụng kiến trúc ResNet-50 kết hợp với cơ chế Attention (CBAM).
		
		%\vspace{0.5cm}
		
		\begin{columns}[T] 
			
			% CỘT TRÁI: TEXT MÔ TẢ
			\begin{column}{0.45\textwidth}
				\begin{block}{Cải tiến chính}
					\begin{enumerate}
						\item \textbf{Module Attention:} Tập trung vào vùng bệnh lý, bỏ qua nền nhiễu.
						\item \textbf{Loss Function:} Sử dụng Focal Loss thay cho Cross-Entropy để xử lý mất cân bằng dữ liệu.
					\end{enumerate}
				\end{block}
				
				\vspace{0.5em}
				\small \textit{*Sơ đồ bên phải minh họa khối Residual Block cơ bản.}
			\end{column}
			
			% CỘT PHẢI: MINH HỌA 
			\begin{column}{0.55\textwidth}
				\begin{alertblock}{Residual Block}
					Đầu ra $y$ của một khối Residual được tính bằng:
					\[ y = \mathcal{F}(x, \{W_i\}) + x \]
					Trong đó:
					\begin{itemize}
						\item $x$: Đầu vào.
						\item $\mathcal{F}$: Hàm mapping dư.
					\end{itemize}
					Việc cộng thêm $x$ giúp giải quyết vấn đề biến mất đạo hàm (vanishing gradient).
				\end{alertblock}
			\end{column}
			
		\end{columns}
	\end{frame}
	
	\useThickBar        
	\useStyleLogoOnly   
	
	\begin{frame}{Kết quả thực nghiệm}
		% Sử dụng môi trường columns để chia cột
		% [T] nghĩa là căn thẳng hàng phía trên 
		\begin{columns}[T]
			
			% --- CỘT TRÁI: BẢNG BIỂU ---
			\begin{column}{0.45\textwidth}
				\begin{block}{So sánh độ chính xác (Accuracy)}
					\centering % Căn giữa bảng trong block
					% Bảng ví dụ cơ bản
					\begin{tabular}{|l|c|c|}
						\hline
						\textbf{Mô hình} & \textbf{Top-1 (\%)} & \textbf{FPS} \\
						\hline
						ResNet-50 (Gốc) & 92.5 & 30 \\
						\hline
						VGG-16 & 89.1 & 24 \\
						\hline
						% Dòng này làm đậm để làm nổi bật kết quả của mình
						\textbf{Đề xuất (Ours)} & \textbf{95.8} & \textbf{45} \\
						\hline
					\end{tabular}
				\end{block}
				
				\vspace{1em} 
				\footnotesize
				\textit{*Số liệu được đo trên tập dữ liệu kiểm thử HUST-D1 (n=1000).}
			\end{column}
			
			
			% --- CỘT PHẢI: HÌNH ẢNH ---
			\begin{column}{0.5\textwidth}
				\begin{figure}
					\centering
					% -----------------------------------------------------------
					% QUAN TRỌNG: Thay 'example-image' bằng tên file ảnh của bạn.
					% Ví dụ: \includegraphics[width=0.95\textwidth]{chart-ket-qua.png}
					% -----------------------------------------------------------
					% Dưới đây là một khung giữ chỗ để demo:
					\fbox{
						\begin{minipage}[c][5cm]{0.9\textwidth}
							\centering \vspace{2cm} \textbf{[CHỖ NÀY ĐỂ ẢNH]}
						\end{minipage}
					}
					% -----------------------------------------------------------
					
					\caption{Biểu đồ trực quan hóa hiệu năng thời gian thực.}
				\end{figure}
			\end{column}
			
		\end{columns}
	\end{frame}
	
	
	% =================================================================
	% PHẦN CUỐI: LIÊN HỆ & CẢM ƠN
	% =================================================================
	
	% Trang liên hệ 
	\hustcontactpage{\hfill THÔNG TIN LIÊN HỆ \hfill }{
		\textbf{Nhóm nghiên cứu:} FRC - HUST \\
		\textbf{Người hướng dẫn:} TS. Nguyễn Văn B \\
		\vspace{0.5cm}
		
		\begin{itemize}
			\item \textbf{Email:} \href{mailto:fami.rebel.club@gmail.com}{fami.rebel.club@gmail.com}
			\item \textbf{Website:} \url{https://www.facebook.com/famirebelclub}
			\item \textbf{Github:} \url{https://github.com/famirebelclub-creator}
		\end{itemize}
		
		\vspace{1cm}
		\textit{Xin chân thành cảm ơn Hội đồng đã lắng nghe!}
	}
	
	% =================================================================
	% TÀI LIỆU THAM KHẢO
	% =================================================================
	
	% Đổi style nền cho thoáng (dễ đọc text nhiều)
	\useStyleLogoOnly 
	
	% Cấu hình icon: Hiển thị số [1], [2] thay vì icon quyển sách mặc định
	\setbeamertemplate{bibliography item}[text]
	
	\begin{frame}[allowframebreaks]{Tài liệu tham khảo}
		
		\begin{thebibliography}{99} % Số 99 nghĩa là căn lề cho danh sách 2 chữ số
			
			% --- TÀI LIỆU TIẾNG VIỆT ---
			\bibitem{toancaocap}
			Nguyễn Đình Trí (Chủ biên), Tạ Văn Đĩnh, Nguyễn Hồ Quỳnh,
			\emph{Toán cao cấp (Tập 1: Đại số và Hình học giải tích)},
			NXB Giáo dục Việt Nam, 2015.
			
			\bibitem{nhapmonai}
			Đinh Mạnh Tường,
			\emph{Giáo trình Trí tuệ nhân tạo},
			NXB Khoa học và Kỹ thuật, 2012.
			
			% --- TÀI LIỆU TIẾNG ANH ---
			\bibitem{resnet}
			K. He, X. Zhang, S. Ren, and J. Sun,
			``Deep Residual Learning for Image Recognition,''
			in \emph{Proceedings of the IEEE Conference on Computer Vision and Pattern Recognition (CVPR)},
			pp. 770–778, 2016.
			
			\bibitem{lecun1998}
			Y. LeCun, L. Bottou, Y. Bengio, and P. Haffner,
			``Gradient-based learning applied to document recognition,''
			\emph{Proceedings of the IEEE}, vol. 86, no. 11, pp. 2278–2324, 1998.
			
			\bibitem{goodfellow}
			I. Goodfellow, Y. Bengio, and A. Courville,
			\emph{Deep Learning},
			MIT Press, 2016.
			
			% --- WEBSITE ---
			\bibitem{pytorch}
			PyTorch Foundation,
			``PyTorch Documentation,''
			\url{https://pytorch.org/docs/stable/index.html},
			truy cập ngày \today.
			
			\bibitem{hustweb}
			ĐH Bách khoa Hà Nội,
			``Quy định về đào tạo đại học,''
			\url{https://hust.edu.vn}, 2024.
			
		\end{thebibliography}
	\end{frame}
	
	
	% Nếu bạn KHÔNG muốn slide Thank You hiện số trang:
	% \hidePageNumber
	
	% Trang kết thúc 
	\hustthankyou
	
\end{document}