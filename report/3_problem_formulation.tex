\section{PROBLEM FORMULATION}

\subsection{Task Definition}
Let $\mathcal{E} = \{1,2,\dots,C\}$ be the emoji inventory with $C=43$ candidates. We are given a dataset
\begin{equation}
    \mathcal{D} = \{(x_i, y_i)\}_{i=1}^{N},
\end{equation}
where $x_i$ is an input text string and $y_i \in \mathcal{E}$ is the emoji observed with the text. We frame emoji prediction as a \textbf{ranking} task: for each query $x$, the model should produce an ordering over emojis such that relevant emojis appear near the top of the list.

\subsection{Model Definition}
A model is a function $f: \mathcal{X} \times \mathcal{E} \to \mathbb{R}$ that scores the compatibility between an input text $x$ and an emoji $y$. Given a query $x$, the model produces a ranking over emojis by sorting them in descending order of their scores $\{f(x,y) : y \in \mathcal{E}\}$. 

\subsection{Evaluation Metrics}
Because multiple emojis can be plausible for a text, we evaluate the \textbf{top-k of the ranking}. Let $\text{Top}^k(x)$ denote the set of $k$ highest-scoring predicted emojis for input $x$. We use \textbf{accuracy at top-k} (Acc@k) as the evaluation metric, defined as the proportion of examples where the ground-truth emoji $y_i$ is among the top-k predictions for input $x_i$:
\begin{equation}
    \text{Acc@}k = \frac{1}{N}\sum_{i=1}^{N} \mathbb{I}\left[y_i \in \text{Top}^k(x_i)\right].
\end{equation}